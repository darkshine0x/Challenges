\documentclass{article}
\usepackage[utf8]{inputenc}
\usepackage[a4paper, total={6.5in, 10in}]{geometry}
\usepackage{amsmath}
\usepackage{fancyvrb}

\title{Frequency Analysis}
\author{Jan Baumann }
\date{March 2021}

\begin{document}

\maketitle

\section{Ziel}
Ziel dieses Projekts ist es ein Programm wirklich end-to-end zu schreiben und zu starten.

\section{Aufgabenbeschreibung}
Es soll ein Programm geschrieben werden, dass ein .txt-File als Input entgegen nimmt und dann die einzelnen Zeichen zählt.

\section{Komponenten}
Das Progamm besteht aus folgenden Teilen:
\begin{itemize}
    \item Klasse, die für das Auszählen der Zeichen verantwortlich ist.
    \item Klasse, die die main-Methode enthält und mit der das Programm gestartet werden kann.
    \item Unit tests, die die Implementierung testen.
\end{itemize}

\end{document}
